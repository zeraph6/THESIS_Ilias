\chapter*{Introduction En Français}

La prolifération rapide des données dans divers domaines et applications d'intelligence artificielle a conduit à une explosion des ensembles de données haute dimension. À mesure que ces collections s'agrandissent, englobant des millions voire des milliards de points de données, le défi réside dans la capacité à traiter, analyser et récupérer efficacement des informations pertinentes. De nombreuses applications axées sur les données, des modèles d'apprentissage automatique aux systèmes de recommandation, dépendent d'une interaction efficace avec ces ensembles pour extraire des insights et des connaissances à la demande, la recherche vectorielle étant une opération centrale.

Bien que les approches traditionnelles de recherche vectorielle, telles que les méthodes basées sur les arbres et le hachage, tentent d'exécuter cette tâche efficacement, elles échouent souvent à offrir des performances satisfaisantes à grande échelle. Au cours de la dernière décennie, les méthodes basées sur les graphes ont émergé comme une solution de premier plan pour une recherche vectorielle efficace, offrant un rappel élevé et une efficacité de requête impressionnante. Cependant, les structures de graphes existantes font face à des défis significatifs lorsqu'il s'agit de s'adapter efficacement à des collections massives.

Cette thèse aborde ces défis en proposant de nouvelles méthodes qui font progresser l'état de l'art en matière de recherche de similarité en haute dimension. Nous introduisons \textit{ELPIS}, une nouvelle approche pour la recherche vectorielle approximative en mémoire qui exploite à la fois les structures d'indexation basées sur les arbres et les graphes, combinant leurs forces pour surmonter leurs limitations mutuelles et surpasser les méthodes de pointe en termes de latence, tout en offrant un débit compétitif. De plus, nous proposons une nouvelle taxonomie qui catégorise les approches basées sur les graphes selon cinq paradigmes clés, fournissant des insights sur les forces et les faiblesses de chacun. Enfin, nous présentons une nouvelle approche basée sur les graphes qui améliore la scalabilité de l'indexation et les performances des requêtes par rapport aux méthodes existantes.

\section*{Aperçu du Problème}

La recherche vectorielle est une opération fondamentale au cœur de nombreuses tâches essentielles d'analyse de données. Elle soutient les systèmes de recommandation~\cite{conf/kdd/wang2018,amazon}, la recherche d'information~\cite{conf/williams2014}, le clustering~\cite{journal/JMLR/bubeck2009,journal/pattrecog/Warren2005}, la classification~\cite{classification,pros}, et la détection d'anomalies~\cite{discord,norma,series2graph,landmines,nba,DBLP:journals/datamine/LinardiZPK20,DBLP:journals/pvldb/BoniolPPF21,DBLP:journals/pvldb/PaparrizosKBTPF22,DBLP:journals/pvldb/PaparrizosBPTEF22} dans divers domaines scientifiques et commerciaux, notamment la bioinformatique~\cite{biof1,biof2}, la vision par ordinateur~\cite{cv1,cv2}, la sécurité~\cite{cybersecurity,cyb2}, la finance~\cite{finance1,finance2}, et la médecine~\cite{medcine1,medicin2}.

Dans l'intégration de données, la recherche de similarité joue un rôle crucial dans la résolution d'entités~\cite{journal/pvldb/ebraheem2018}, le remplissage de valeurs manquantes~\cite{retro} et la découverte de données~\cite{journal/pvldb/zhu2016}. De plus, elle est largement utilisée en ingénierie logicielle~\cite{journal/pacml/uri2019,conf/icsec/nguyen2016} pour surveiller l'utilisation des E/S et automatiser les mappages d'API, ainsi qu'en cybersécurité pour profiler l'activité réseau et détecter les logiciels malveillants~\cite{cybersecurity,cyb2}. Plus récemment, la recherche de similarité est devenue de plus en plus vitale pour améliorer les performances et l'interprétabilité des grands modèles de langage (LLM), aidant à réduire les hallucinations dans le contenu généré~\cite{retrieval-diffusion-models,dense-passage-retrieval,seq2seq,rag-nlp,rag0,rag1,rag2,rag3}.

Le problème de la recherche vectorielle a été largement étudié au cours des 30 dernières années~\cite{hnsw,hercules,rng,hydra1,hydra2} sous diverses terminologies, y compris la recherche de similarité, ou souvent réduit au problème des plus proches voisins (\textit{k}-NN)~\cite{conf/icde/echihabi2021,conf/sigmod/echihabi2020,gogolou2019progressive}. À mesure que les collections de données haute dimension continuent de croître à des rythmes sans précédent, le besoin de solutions de recherche vectorielle efficaces et optimisées a attiré une attention croissante.

Ce problème peut être abstrait comme la recherche de l'objet ou des objets les plus similaires dans une collection d'objets basée sur une mesure de similarité, tels que les \textit{k} plus proches voisins~\cite{conf/sigmod/echihabi2020, aumuller2017ann,ann-benchmark-journal}. Ces objets peuvent représenter du texte, des images, des vidéos, des graphes, des séries de données, des tables de bases de données ou des représentations apprises, qui sont finalement encodées sous forme de vecteurs dans un espace vectoriel~\cite{hydra1,hydra2,DBLP:conf/edbt/EchihabiZP21,conf/sigmod/echihabi2020}. La mesure de similarité, telle que la distance euclidienne~\cite{euclid}, est utilisée pour identifier les objets dont les représentations sont les plus proches d'un objet de requête donné.

Différentes variantes de ce problème existent. En fonction du niveau de précision requis, une recherche de similarité peut être exacte, où l'objectif est de retourner exactement les objets les plus proches de la requête à partir du jeu de données~\cite{hercules,hydra1,conf/icde/echihabi2021,messi,parisplus,dumpy,dpisax,kdtree,dstree,isax2+,ulisse,vafile,twinsubsequences,oddysey,dstree}, ou approximative, où la précision est échangée contre des réponses plus rapides et moins gourmandes en ressources~\cite{hydra2,hercules,qalsh,kdtree,kgraph,efanna,hnsw,dpg,conf/icassp/jegou2011,journal/iccv/xia2013,journal/pami/babenko15,hnsw,hcnng,nsg,vamana}. Le problème approximatif peut être classé en deux catégories principales : la recherche de similarité approximative avec des garanties sur les bornes d'erreur~\cite{conf/vldb/lv2007,sk-lsh,journal/pvldb/zheng2020,journal/pvldb/zhu2016,conf/stoc/indyk1998,conf/vldb/sun14,qalsh,hydra2,srs,conf/sigmod/gogolou20}, et la recherche de similarité approximative sans garanties sur les bornes d'erreur, appelée \textit{ng}-approximate, où les performances plus rapides sont privilégiées par rapport aux bornes d'erreur~\cite{kdtree,hydra2,elpis,hercules,hnsw,nsg,hcnng,efanna,nsg,nssg,nsw11,vamana,ieh,dpg,kgraph}.

Avec la récente montée des applications d'IA~\cite{rag0, nsg,alibabaknngml, recommender_systems,faiss,amazon}, la recherche de similarité \textit{ng}-approximate a gagné en attention, en particulier puisque de nombreuses applications d'IA n'exigent pas de réponses exactes pour les requêtes \textit{k}-NN. Ces applications peuvent obtenir des résultats satisfaisants avec une précision moyenne, garantissant des temps de réponse raisonnables. Des exemples incluent les systèmes de recommandation~\cite{conf/kdd/wang2018,amazon,nsg}, les moteurs de recherche d'images~\cite{nsg,faiss}, et les modèles de génération augmentée par récupération (RAG)~\cite{retrieval-diffusion-models,dense-passage-retrieval,seq2seq,rag-nlp}, qui combinent de grands modèles de langage avec des moteurs de recherche vectoriels pour récupérer efficacement un contexte pertinent~\cite{retrieval-diffusion-models,rag-nlp,rag0,rag1,rag2,rag3}. Cette intégration aide à générer des réponses de requête plus précises et à jour, renforçant le rôle critique de la recherche vectorielle dans les applications d'IA modernes.

\section*{Contributions Principales}

\textbf{1. ELPIS : Une Recherche Vectorielle Basée sur les Graphes et Évolutive en Mémoire}

Cette thèse introduit \textit{ELPIS}, un nouveau cadre pour la recherche vectorielle approximative en mémoire qui combine les forces des structures d'indexation basées sur les arbres et les graphes. En tirant parti de l'organisation hiérarchique des arbres et de la sommation de l'EAPCA-Tree pour la scalabilité, ainsi que de l'efficacité des méthodes basées sur les graphes pour la latence des requêtes, ELPIS aborde les lacunes des deux approches. Il réduit significativement la latence des requêtes tout en maintenant une précision compétitive, même dans de grands ensembles de données comprenant des milliards de vecteurs haute dimension. Cette approche permet à ELPIS de s'adapter efficacement à l'augmentation des tailles de données, le rendant adapté aux applications modernes axées sur les données qui nécessitent à la fois une grande vitesse et une précision élevée. Dans des évaluations approfondies, ELPIS surpasse systématiquement les méthodes de pointe existantes, en particulier dans les environnements sensibles à la latence.

\textbf{2. Enquête Complète et Taxonomie des Méthodes de Recherche Basées sur les Graphes}

Une contribution majeure de ce travail est une enquête détaillée sur les techniques de recherche de similarité basées sur les graphes. Cette enquête offre une évaluation systématique des méthodes actuelles et propose une nouvelle taxonomie qui catégorise ces approches selon cinq paradigmes clés : \textit{Sélection de Graines (SS)}, \textit{Propagation de Voisinage (NP)}, \textit{Insertion Incrémentale (II)}, \textit{Diversification de Voisinage (ND)} et \textit{Diviser pour Régner (DC)}. La taxonomie fournit un cadre organisé pour comprendre la variété des méthodes basées sur les graphes, clarifiant leurs forces et limitations respectives. Cette contribution offre aux chercheurs et aux praticiens des insights précieux pour sélectionner la méthode la plus appropriée pour des applications spécifiques, tout en identifiant des domaines pour de futures explorations. Cette enquête sert de référence fondamentale pour la communauté, soulignant les défis de la scalabilité et de l'optimisation de la recherche de similarité basée sur les graphes pour des ensembles de données de plus en plus grands et complexes.

\textbf{3. Optimisation du Débit via la Fusion Basée sur l'EAPCA}

Cette thèse introduit une nouvelle technique de fusion basée sur l'EAPCA qui aborde efficacement les limitations de débit d'ELPIS. En utilisant les distances de bornes inférieures de l'Extended Adaptive Piecewise Constant Approximation (EAPCA), le processus de fusion combine intelligemment de plus petits graphes optimisés pour la latence en de plus grandes structures optimisées pour le débit. Cette approche de fusion sélective réduit significativement les calculs de distances redondants lors du traitement des requêtes, permettant une traversée plus efficace de l'espace vectoriel. En conséquence, ELPIS est capable d'équilibrer efficacement les performances optimisées pour la latence et le débit, fournissant une solution robuste pour les tâches de recherche vectorielle à grande échelle sans nécessiter la construction d'index séparés optimisés pour chaque scénario.

\textbf{4. OIGAS : Graphes d'Insertion Incrémentale Optimisés pour une Recherche Vectorielle Efficace}

En s'appuyant sur les conclusions de l'enquête, cette thèse propose un nouvel algorithme de recherche basé sur les graphes conçu pour améliorer à la fois la scalabilité de l'indexation et l'efficacité des requêtes. La nouvelle approche optimise l'indexation des graphes en réduisant le calcul requis pour construire l'index de manière incrémentale, ainsi qu'en proposant des stratégies combinées pour l'élagage des arêtes dans le graphe basées sur le degré entrant des nœuds, permettant de gérer des ensembles de données volumineux plus efficacement que les techniques existantes. Cette avancée offre une amélioration significative par rapport aux méthodes traditionnelles, notamment dans les scénarios qui nécessitent une indexation rapide de grandes collections de données avec une perte minimale de qualité des réponses lors de la recherche.

%\section*{Conclusion et Perspectives Futures}

%En résumé, cette thèse a exploré et amélioré les méthodes basées sur les graphes pour la recherche approximative des plus proches voisins. En combinant les forces de différents paradigmes et en introduisant de nouvelles approches, nous avons adressé les défis de scalabilité et d'efficacité dans la recherche vectorielle à grande échelle.

%\textbf{Perspectives Futures :} Les travaux futurs incluront l'extension d'ELPIS à des environnements distribués et hors mémoire, l'utilisation de l'apprentissage automatique pour l'ajustement adaptatif des requêtes, et le développement de techniques avancées de sélection de graines et de diversification du voisinage pour améliorer davantage les performances de recherche.

